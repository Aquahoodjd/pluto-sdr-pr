% !TeX root = ../report.tex
\chapter{Einleitung}

Vorliegend ist eine Studienarbeit zur Implementierung eines Passivradar-Systems mittels LTE-basiertem 5G Multimedia-Broadcast Beleuchter. Das Projekt umfasst Konzeption, Beschaffung, Konstruktion, Testen, sowie Analyse eines ADALM-PlutoSDR basierten Sensorsystems zur passiven kohärenten Detektion startender und landender Verkehrsflugzeuge aus mehreren Kilometern Distanz. Angestrebt wird dabei ein non-kooperatives Detektionsverfahren, in dem das Multimedia-Broadcast Signal eines regional betriebenem Funkturms des Südwestdeutschen Rundfunks zur Zielbeleuchtung dient. Neben der praktischen Umsetzung wird in einer wissenschaftlichen Vertiefung auch auf theoretische Aspekte der Passivradar-Technologie eingegangen. Dabei sollen Technologiegrundlagen, sowie eine genauere Betrachtung des Beleuchtersignals vorgestellt werden.

\section{Projektziel}

Ziel der Projektarbeit ist, die Aufbereitung und Verarbeitung der Referenz- und Echosignale, um visuelle Detektion eines im Betrachtungsgebiet befindlichen Verkehrsflugzeug in einer Range-Doppler Matrix zu erlauben. Darüber hinaus soll keine maschinelle Alarmgenerierung, kartesische Positionsbestimmung oder Tracking der Ziele erfolgen. Die Herausforderungen dieser Projektarbeit liegen in der Umsetzung eines Systems mit einer realen unabhängig gesteuerten 5G Broadcast Sendeanlage. Somit sind Signalparameter bis auf die in~\cite{5GMAG2020} geschilderte Rahmendaten weitgehend unbekannt und müssen durch Annahmen und Nachmessungen ergänzt werden.

\section{Vorangegangene Arbeit}

Passive Radarsysteme sind in der Literatur seit langer Zeit bekannt und erfreuen sich neuerdings wiederauferlebtem Interesse. Besonders im militärischen Einsatz bietet Passivradar potenziell taktische Vorteile. Ohne eigene Abstrahlungen lässt es sich zur verdeckten Überwachung des Luftraums nutzen. Wenn eingesetzt als Ergänzung zu aktiven Radarsystemen, die nur zur Zielaufschaltung eingeschaltet werden, lässt sich die Bedrohung durch radarsuchenden Raketen oder Drohnen minimieren. Auch über die zivile Nutzung zur weiträumigen Luftüberwachung wird nachgedacht~\cite{Stahl2018}. In zahlreichen Publikationen werden Systeme mit FM~\cite{Lallo2008,Xie2018}, DAB~\cite{Winkler2021} oder DVB-T/2~\cite{Conti2016,Winkler2017} Beleuchtern untersucht. Eine handvoll Hersteller bieten bereits produkttaugliche Multiband-Systeme an~\cite{Lutz2018}. Die theoretische Betrachtung LTE-basierter Multimedia-Broadcast-Signale als Beleuchter wird unter anderem in~\cite{Klöck2019} diskutiert.
